\documentclass[12pt, a4paper, titlepage, oneside]{article}

\usepackage{times}
\usepackage[brazil]{babel}
\usepackage{fancyhdr}
\usepackage[utf8]{inputenc} %codificacao
\usepackage[T1]{fontenc}
\usepackage{graphicx} %incluir figuras
\usepackage{subcaption}
\usepackage{hyperref}
\usepackage{indentfirst} %identa o primeiro parágrafo

%\usepackage{amssymb}
\usepackage{amsmath}
\usepackage{amsfonts}
\usepackage[left=2.5cm,right=2.5cm]{geometry}

%\title{Relatório}
%\author{Grupo 5}

\begin{document}	
	\begin{titlepage}
		\begin{center}
			\textbf{ESCOLA POLITÉCNICA DA UNIVERSIDADE DE SÃO PAULO\\
			\begin{figure}[h!]
				\centering
				\includegraphics{./imagens/minerva_top.png}
				\label{fig:minerva_top}
			\end{figure}
			PSI-3214 - Laboratório de Instrumentação Elétrica\\\vspace{1.4cm}
			\begin{figure}[!h]
				\centering
				\includegraphics[scale=0.7]{./imagens/harmony.png}
				\label{fig:harmony}
			\end{figure}
			%Projeto Harmony\\\vspace{2cm}
			Turma 5 - Grupo 5\\\vspace{0.5cm}
			%\begin{description}
			%	\centering
			%	\item[Filipe Gabriel Santos] 9837761
			%	\item[Renato de Oliveira Freitas] 9837708
			%	\item[Tamy Beatriz Miyano] 9737563
			%\end{description}
			Filipe Gabriel Santos........................9837761\\
			Renato de Oliveira Freitas................9837708\\
			Tamy Beatriz Miyano........................9837563\\\vspace{2.6cm}
			\today}
		\end{center}
	\end{titlepage}
	
	\tableofcontents
	\newpage
	
	\begin{abstract}
		
			Ondas sonoras consistem em vibrações mecânicas de um meio natural e se propagam com uma velocidade que depende de diversos fatores, entre eles a natureza do meio. 
			
Nota musical é um termo empregado para designar o elemento mínimo de um som, formado por um único modo de vibração do ar. Sendo assim, a cada nota corresponde uma duração e está associada uma frequência. Apesar de diferentes, quando os instrumentos reproduzem uma mesma nota, o som é agradável. Isso acontece pois todas as componentes dessa melodia são compostas por múltiplos inteiros da frequência original. Para entender melhor esse conceito podemos analisar a tabela ao lado, que relaciona as notas e suas respectivas frequências: A nota dó (C) corresponde a frequência de 261,63 Hz, enquanto a nota sol (G) é aproximadamente 392 Hz, que é 3/2 a frequência do dó. \cite{nota-musical}

Outro fator importante para diferenciar os sons é a amplitude de cada harmônica, ou seja, sua intensidade. Existem instrumentos em que algumas harmônicas têm amplitudes superiores a própria frequência fundamental, tornando o som bem mais grave ou agudo.

Matematicamente, representa-se o espectro sonoro como uma série de Fourier, uma função no domínio das frequências, em oposição à forma de onda que é uma função no domínio do tempo. Qualquer onda sonora, assim como qualquer outro fenômeno ondulatório, pode ser representado através de seu espectro. Um gráfico de espectro sonoro é composto de barras, cada uma delas representando a amplitude de uma das frequências componentes do som analisado. Este tipo de gráfico é utilizado em equipamentos eletrônicos, tais como analisadores de espectro ou em equalizadores. No caso dos analisadores digitais, o cálculo é realizado através da Transformada Rápida de Fourier - FFT (Fast Fourier Transform), um algoritmo bastante eficiente que permite calcular o valor de uma transformada discreta de Fourier, em tempo real. \cite{espectro-sonoro}

Com base nisso, o projeto consiste em desenvolver um detector de notas musicais utilizando sensores acoplados ao condicionador de sinais, conversor AD e software de monitoramento, no caso, o LabVIEW \cite{labview}, capaz de identificar uma sequência de 5 notas musicais monofônicas.
	
	\end{abstract}
	
	\section{Introdução}
		O projeto de 2017 da disciplina de Laboratório de Instrumentação Elétrica teve como tema o desenvolvimento de um instrumento virtual de monitoramento capaz de adquirir, processar e  visualizar graficamente sinais de sensores. O objetivo deste projeto foi de que, ao longo do semestre,  o grupo aplicasse os conhecimentos adquiridos em aula para a composição do sistema.
O grupo desenvolveu um produto que, com o uso de microcontroladores, transferisse leituras analógicas para uma unidade de processamento digital, como um PC, e aplicasse as transformações de Fourier e outras ferramentas matemáticas. 
Para os testes iniciais, foi utilizado um programa online que gerasse frequências programadas. Mas o instrumento utilizado foi  um teclado arranjador cujas notas foram transmitidas ao circuito de condicionamento. O sinal obtido pelo circuito foi adquirido pelo microcontrolador,  no caso, o Arduino Uno, e processado para que o programa LabVIEW lesse as informações e apresentasse uma interface gráfica.
		
		Abaixo, apresenta-se um diagrama com os elementos do sistema:
		\begin{figure}[h!]
			\centering
			\includegraphics[scale=0.65]{./imagens/diagrama_blocos.png}
			\label{diagrama}
			\caption{Diagrama de blocos dos elementos do sistema}
		\end{figure}

 	\pagebreak
 	
 	\section{Projeto}
 		\subsection{Aquisição de sinais}
 			Neste item, é descrito como o som do teclado será adiquirido e processado pelo Arduino.
 			
 			Para a composição do sistema, foram utilizados os componentes listados abaixo:
 			\begin{itemize}
 				\item Amplificador Operacional LM 741
 				\item Duas baterias de 9V
 				\item Dois conectores para baterias de 9V
 				\item Potenciômetro de 10k$\Omega$
 				\item Três resistores de 100k$\Omega$
 				\item Capacitor eletrolítico de10$\mu$F
 				\item Capacitor cerâmico de 47nF
 				\item Arduino UNO
 				\item Protoboard
 				\item Cabo P2
 				\item Plug fêmea P2
 			\end{itemize}
 			
 			Um sinal de áudio é composto por milhões de pequenas oscilações. Quando relacionado à eletrônica, essas oscilações representam tensões oscilatórias.
 			
 			Quando um sinal de áudio é observado no osciloscópio, é possível notar que a onda oscila em volta da linha de 0V. A amplitude de um sinal de áudio é a distância entre a tensão central e o pico mais alto, ou mais baixo. Por exemplo, num osciloscópio, se a amplitude for de 3V, o máximo será +3V e o mínimo, -3V. O problema é que o Arduino só consegue medir tensões entre 0 e 5V. Se for realizada uma medição de tensão abaixo de 0V, a parte negativa da onda seria cortada. Portanto, idealmente, seria necessário buscar obter uma onda de amplitude 2,5V que oscile por volta de 2,5V.
 			
 			Sinais gerados por instrumentos musicais são obtidos com amplitudes muito baixas, por isso precisam ser amplificadas para a amplitude desejada, no caso, 2,5V, com um aplificador operacional. 
 			
 			Entretanto, o centro da onda de tensão ainda estará em 0V e, para corrigir esse problema, o offset deverá ser alterado. Assim, colocando um offset de 5V no Arduino, o centro de tensão será 2,5V ao invés de 0V, e então a onda não cairá abaixo de 0V.
 			\pagebreak
 			\begin{figure}[h!]
 				\centering
 				\includegraphics[scale=0.5]{./imagens/esquematico.png}
 				\label{fig:esquematico}
 				\caption{Circuito de condicionamento com amp op e offset de 5V}
 			\end{figure}
 			
 			Para obter o sinal de áudio do instrumento, foi utilizado um cabo macho P2 e um plug P2 fêmea para a conexão no circuito.
 			
 			Além disso, serão utilizadas duas baterias de 9V conectadas em série para criar uma alimentação simétrica de +9V e -9V. As alimentações positiva e negativa serão então conectadas no amp op. O sinal obtido da fonte de áudio pelo cabo é ligado à entrada não inversora. Da junção das baterias de 9V em série, faz-se uma ligação ao terra. Para a realimentação, coloca-se um resistor de 100k$\Omega$ entre a saída e a entrada inversora do amp op. Para ajustar o ganho do amp op, será utilizado um potenciômetro de 10k$\Omega$
 			
 			A seguinte equação descreve a relação entre a amplitude do sinal antes e depois de passar pelo amp op:
 			
 			\begin{equation}
 				\frac{V_{out}}{V_{in}} \approx 1 + \frac{R2}{R1}
 			\end{equation}
 			
 			No caso, $R2 = 100k\Omega$ e $0 \leq R1 \leq 10k\Omega$. Ajustando o potenciômetro, é possível ajustar o ganho de amplificação.
 			
 			O circuito DC offset tem um divisor de tensão e um capacitor. O divisor de tensão possui dois resistores de 100k$\Omega$ em série que ligam a alimentação de 5V do Arduino ao terra. Como os resistores têm as mesmas resistências, a tensão na junção entre eles é igual a 2,5V. Essa junção de 2,5V é ligada à saída do amp op através de um capacitor de 10$\mu$F. Como a tensão no lado do amp op aumenta e diminui, isso faz com que a carga acumule momentaneamente e seja repelida para o lado do capacitor conectado à junção de 2,5V. Isso faz com que a tensão na junção de 2,5V oscile centrada em  2,5V, para cima e para baixo.
 			
 			\begin{figure}[h!]
 				\centering
 				\includegraphics[scale=0.4]{./imagens/protoboard}
 				\label{fig:protoboard}
 				\caption{Ilustração do circuito montado na protoboard}		
 			\end{figure}
 			
 			Em relação à etapa 2, o painel frontal foi modificado porque foram encontrados problemas ao realizar a etapa 3. Assim como foi descrito anteriormente, cria-se um vetor no VI que recebe todas as informações obtidas pelo Arduino e, com esses dados, é calculada a frequência predominante e os gráficos da onda e espectro de frequências. Porém, com a extensão LINX, não era possível identificar mais de uma nota musical, pois ela ocupava mais da metade da memória do Arduino, impossibilitando a aquisição de pontos numa frequência aceitável. Desse modo, essa extensão não foi utilizada. Em vez disso, para aumentar a frequência de amostragem, e consequentemente uma resolução de onda melhor definida, é criado um vetor de 300 valores de 1 byte, preenchidos com a leitura do conversor analógico-digital do Arduino, originalmente de 10 bits e convertidos para 1 byte, enviado serialmente para o LabView.
 			
 			\begin{figure}[h!]
 				\centering
 				\includegraphics[scale=0.6]{./imagens/programa_arduino}
 				\label{fig:programa}
 				\caption{Algorítmo para aquisição de dados}
 			\end{figure}
 			\pagebreak
 			
 			Para a comunicação serial, é utilizada a estrutura VISA. Inicializa-se a porta serial desejada, o programa entra num laço onde são realizadas as leituras dos dados enviados.
 			
 			\begin{figure}[h!]
 				\centering
 				\includegraphics[scale=0.9]{./imagens/vi_inicializacao}
 				\label{fig:inicializacao}
 				\caption{Inicialização da comunicação serial}
 			\end{figure}
 			
 			A estrutura retorna os dados lidos na forma de \textit{string}, portanto são necessárias conversões: primeiro, essa \textit{string} é transformada num vetor de valores de 1 byte (0 a 255); desses valores, é subtraído 130 para que fiquem centrados no 0 (-130 a 125), transformados em \textit{double} e, finalmente, multiplicados por 0,02 para que resultem valores de -2,6 a +2,5, representando a amplitude real do sinal adquirido.
Os valores são concatenados em um vetor final, que armazena todo o histórico de dados desde o início do programa. Com esse vetor forma-se o gráfico do sinal, com pontos espaçados em 0,0001s, pois a frequência de amostragem do Arduino é igual a 10 KHz.

 			\begin{figure}[h!]
				\centering
				\includegraphics[scale=0.8]{./imagens/vi_aquisicao}
				\label{fig:aquisicao}
				\caption{Aquisição e conversão de dados} 			
 			\end{figure}
 			
 			Com esse vetor forma-se o gráfico do sinal, com pontos espaçados em 0,0001s, pois a frequência de amostragem do Arduino é igual a 10 KHz. Posteriormente, é realizada sua FFT, resultando num segundo gráfico, o do espectro das frequências, em que é possível detectar a frequência predominante e identificar a nota musical tocada.
 			
 			\begin{figure}[h!]
 				\centering
 				\includegraphics[scale=0.7]{./imagens/vi_graficos}
 				\label{fig:graficos}
 				\caption{Gráficos de amostragem do sinal}
 			\end{figure}
 			
 		\subsection{Interface gráfica}
 			A interface gráfica foi elaborada com o intuito de ser prática e visualmente agradável. O gráfico de baixo é a amostragem de sinal com espaçamento de 0,0001s entre cada ponto. O de cima é resultado da FFT, o espectro de frequências, da qual é possível identificar a frequência predominante. Ao lado, mostra-se a lista de notas musicais possíveis de serem identificadas e, logo abaixo, uma imagem de uma nota musical que sinaliza qual é a nota captada. 
 			
 			\begin{figure}[h!]
 				\centering
 				\includegraphics[scale=0.4]{./imagens/front_panel1}
 				\label{fig:front_panel}
 				\caption{Interface gráfica do painel frontal}
 			\end{figure}
 			
 	\section{Resultados}
 		Depois de muitos testes realizados com o circuito provisório montado em protoboard, foi construído o mesmo sistema em uma placa com todos os componentes soldados. 
 		
 		\begin{figure}[h!]
 			\centering
 			\includegraphics[scale=0.2]{./imagens/circuitos}
 			\label{fig:circuito_soldado}
 			\caption{Circuito definitivo e conversor analógico-digital}
 		\end{figure}
 		\pagebreak
 		
 		Para que o sistema não apresente falhas de estrutura e danos mecânicos, foi construída uma caixa para protegê-lo de tais hipotéticas situações.
 		
 		\begin{figure}[!h]
 			\centering
 			\includegraphics[scale=0.17]{./imagens/caixinha}
 			\label{fig:caixinha}
 			\caption{Compartimento para proteção mecânica do circuito}
 		\end{figure}
 		
 		
 		Realizados os testes e ajustado o VI, conseguiu-se obter um sistema que identificasse uma sequência infinita de notas musicais, e não apenas uma sequência de cinco notas.
 		
 		 		
 		\begin{figure}[h!]
			\centering
			\begin{subfigure}{.5\textwidth}
				\centering
				\includegraphics[scale=0.085]{./imagens/pc1}
				\label{fig:pc1}				
			\end{subfigure}%
			\begin{subfigure}{.5\textwidth}
				\centering
				\includegraphics[scale=0.085]{./imagens/pc2}
				\label{fig:pc2}				
			\end{subfigure}
			\caption{Testes realizados com o sistema pronto utilizando uma
				fonte de frequência contínua}
		\end{figure}
		\pagebreak
 		
 		Testes posteriores com mais de uma nota tocada simultaneamente também foram feitos. No caso, foram transmitidos sinais senoidais de 220 Hz, 440 Hz e 880 Hz, sendo a segunda com volume maior, ou seja, maior amplitude. O resultado foi uma amostragem de sinal não senoidal, pois foi uma soma das senoides de cada frequência. Além disso, o gráfico do espectro de frequências também demonstrou que o sistema é capaz de identificar sinais separados. O mais importante do teste foi identificar a frequência predominante, ou seja, a de maior amplitude, que foi a de 440 Hz (no caso, o valor foi aproximado devido às interferências dos outros sinais).
 		
 		\begin{figure}[h!]
 			\centering
 			\includegraphics[scale=0.4]{./imagens/nao_senoidal}
 			\label{fig:nao_senoidal}
 			\caption{Soma de três ondas senoidais diferentes}
 		\end{figure}

		\pagebreak
		
	\section{Conclusão}
	A complexidade de se alinhar os conteúdos teóricos com uma aplicação prática (afinador) foi bastante desafiadora. 
	
Nesta última etapa, foram encontrados vários problemas a serem solucionados somente após muita pesquisa e experiência com o programa LabView. 

No começo, a maior dificuldade era a familiaridade com a teoria a ser aplicada no VI, como por exemplo, a TDF e FFT. Ao longo do desenvolvimento do projeto, o leque extenso de ferramentas do LabView foi responsável pelas muitas horas gastas para entender o programa e saber construir o código mais viável. 
Com isso, foi possível perceber mais uma vez a importância de um sólido conhecimento sobre as ferramentas a serem utilizadas, mais a teoria a ser aplicada, para que um problema seja resolvido com êxito. 

Agradecemos especialmente aos professores que nos transmitiram conhecimento, aos monitores que se mostraram dispostos a ajudar e aos amigos e veteranos com quem houve trocas de ideias. 

	\pagebreak
	
	\bibliographystyle{ieeetr}	
	\bibliography{References}		

\end{document}